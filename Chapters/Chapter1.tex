% Chapter 1

\chapter{Introduction} % Main chapter title

\label{Chapter1} % For referencing the chapter elsewhere, use \ref{Chapter1} 

\lhead{Chapter 1. \emph{Introduction}} % This is for the header on each page - perhaps a shortened title

%----------------------------------------------------------------------------------------

\section{An Easier Biomedical Research}

I'll try to present the main idea of this project in following story. Imagine you in the position as a biomedical researcher, are currently investigating some unknown enzymes that somehow were over-expressed in a patient with medical conditions. Upon some more investigating, you managed to get the sequence of several proteins. Without prior knowledge of the proteins, you would naturally BLAST the sequences and wait a little while while the BLAST is searching the sequence against your local database or some online service. Upon the results were coming, you would naturally want to check the resulting proteins one by one, at least the best matching ones. For each protein, you would want to search for articles that have dealt with this protein before.

Imagine that, instead of having to go through blasting the sequence manually and searching for articles one by one, you could just put in a sequence in a website, wait for a while and get the site returns a list of articles that mention the proteins with similar or exact sequence to the one you have. Not only you would save time and resource during the parts that were handled by website itself, you as a researcher could focus more on the substantial part of the research -- that is, finding as much essential information about the unknown protein in as little overhead as possible. Therefore, I created a web service that realizes this. In the service, user would only have to put in the sequence of unknown protein, press the search button and receive at the end a list of articles that mention proteins with identical or similar sequence to queried proteins.

With this small contribution, we hope not only to bridge the gap between sequence and knowledge discovery in biomedical research, but also give researcher more flexibility and insights in their literature research. With also ongoing feature extensions and updates, we would also hope that the service would serve more researchers with more conveniences both in medium and long run.

\section{BLAST}

As the title of this thesis already conveyed, the main idea of this project is to bridge the accessibility and knowledge gap between sequence and the main source of knowledge and reference of previous discoveries -- a vast corpora of publications in natural sciences -- through a modern search engine. Given a sequence of amino acids, it would be impossible for a human to directly identify directly the protein, let alone the characteristics and the functions and the characteristics of the protein.

Several attempts on bridging one component of the gap was done in eighties and earlier nineties. In 1985, Lipman and Pearson published the first paper mentioning the DNA and protein sequence alignment program FASTA \citep{Lipman85}

Further down the road, in 1990, Altschul et al. published the Basic Alignment Research Tool \citep{Altschul90}

%----------------------------------------------------------------------------------------
