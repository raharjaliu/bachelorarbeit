% Chapter 1

\chapter{Introduction} % Main chapter title

\label{Chapter1} % For referencing the chapter elsewhere, use \ref{Chapter1} 

\lhead{Chapter 1. \emph{Introduction}} % This is for the header on each page - perhaps a shortened title

%----------------------------------------------------------------------------------------

\section{An Easier Biomedical Research}

We'll try to present the main idea of this project in following story. Imagine you in the position as a biomedical researcher, are currently investigating some unknown enzymes that somehow were over-expressed in a patient with medical conditions. Upon some more investigating, you managed to get the sequence of several proteins. Without prior knowledge of the proteins, you would naturally BLAST the sequences and wait a little while while the BLAST is searching the sequence against your local database or some online service. Upon the results were coming, you would naturally want to check the resulting proteins one by one, at least the best matching ones. For each protein, you would want to search for articles that have dealt with this protein before.

Imagine that, instead of having to go through blasting the sequence manually and searching for articles one by one, you could just put in a sequence in a website, wait for a while and get the site returns a list of articles that mention the proteins with similar or exact sequence to the one you have. Not only you would save time and resource during the parts that were handled by website itself, you as a researcher could focus more on the substantial part of the research -- that is, finding as much essential information about the unknown protein in as little overhead as possible. Therefore, we created a web service that realizes this. In the service, user would only have to put in the sequence of unknown protein, press the search button and receive at the end a list of articles that mention proteins with identical or similar sequence to queried proteins.

With this small contribution, we hope not only to bridge the gap between sequence and knowledge discovery in biomedical research, but also give researcher more flexibility and insights in their literature research. With also ongoing feature extensions and updates, we would also hope that the service would serve more researchers with more conveniences both in medium and long run.

\section{Overview of This Thesis}

In this thesis, we will describe how we came with the idea of creating PubSeq, how we did that and what we planned in the future regarding our implementation.

In \textbf{Chapter 2}, we will discuss how bridging the knowledge gap has been attempted in the past and how our contribution would fit in the bigger picture. We also discuss some of the methods that are relevant in our project. Also, we would look into how our project builds upon existing knowledge and technology.

\textbf{Chapter 3} introduces the system as a whole. How we organize the sub-components together. We would also skim through the technological side of the projects here, while keeping the reader aware of the bigger picture. We would discuss our rationale behind selecting some of technology stacks that we used. All the while,we would also show some the visual examples from our component here. By the end of the chapter it is hoped that the reader would understand how each single component interacts with others within our system.

\textbf{Chapter 4} explains in detail our Tagging Pipeline. Here the reader would see how the MEDLINE corpus would be processed, annotated and pushed onto Solr Index.

\textbf{Chapter 5} explains our storage technology, the Solr index. Besides the technology itself, we would explain how we configure and structure our index that would fit the resulting data from Tagging Pipeline.

\textbf{Chapter 6} investigates our web server component. We would first present the technological stacks that were used in this component and other specifications. We would then present the program from the perspective of end user. We would show how convenient would that be for a researcher to use our application, which would make our case for value proposition of PubSeq search engine.

\textbf{Chapter 7} covers quantitative measurement of the quality of our website. We would focus mostly on how our system performs, especially with regards to the sensitivity and specificity. We would focus mostly on the the quality of protein tagging within our data (see Chapter \ref{Chapter5} for detail). We would also muse on how our system would have an edge over similar UniProt ID-based abstract search service provided by UniProt \citep{uniprot2008universal} \citep{Uniprot2011} \citep{UniprotOnline}.

%\textbf{Chapter 6} explains our update and maintenance design for the website. Here the reader would be aware on how we attempt to make our website up-to-date to the latest protein landscape. We would, again, delve into how this is realized within our system.

\textbf{Chapter 8} covers our conclusion of the system so far. There we presented our own ideas on how the website could and would be improved. we would again reiterate the the merits of using the PubSeq as the search engine for abstracts based on protein sequence.

%----------------------------------------------------------------------------------------
