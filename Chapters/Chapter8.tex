% Chapter Template

\chapter{Conclusion and Outlook} % Main chapter title

\label{Chapter8} % Change X to a consecutive number; for referencing this chapter elsewhere, use \ref{ChapterX}

\lhead{Chapter 7. \emph{Conclusion and Outlook}} % Change X to a consecutive number; this is for the header on each page - perhaps a shortened title

In this project we tried to tackle issues arising from two contemporary development in modern bioinformatics:

\begin{itemize}
\item Explosion of the knowledge written information corpus in biomedical research. This is reflecting by the staggering number of publications in biomedical fields which clock at almost 25,000,000 publications at the moment (see previous chapters).

\item Explosion in the number of discovered proteins, especially the automatically annotated non-reviewed proteins.
\end{itemize}

In this project we tried to address the two source of problems seen from the perspective of a human, a researcher. We managed so far to create a working system that address some of the issues posed in Chapter \ref{Chapter1}. Not only working, we design our system to be self-updating especially with more additions of references within MEDLINE corpus. While a perfect portability is practically impossible, we design our Pipeline in a way that it would make extension easier than it would given the system complexity. In administrative side, we also have created an internal developer wiki which would make continuous development possible. Thus we hope that this effort wouldn't stop. And even if it is, we expect the system to keep running and stay updated as we already designed -- this would ensure that our project would stay on helping more researcher in their research later.

Our evaluation shows a promising results with regard to our NER Tagger that we deploy in our system. This is great news since we can be confident that our results are based on robust system. Thus we hope that more researcher could make use of our tool.

The nature of our project means that our system will keep evolving for some time. This would mean that some small things might change or be replaced, for good. We however, we will keep emphasize the importance of both simplicity and intuitive design, particularly regarding our web interface, which will ensure our system usability among broad spectrum of user (from undergraduate students to senior principal investigator). Administratively, we wish to continue this project actively until early part of next year as Master Project. As thus, there would be more updates coming in this project, which we are excited about.