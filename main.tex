%%%%%%%%%%%%%%%%%%%%%%%%%%%%%%%%%%%%%%%%%
% Masters/Doctoral Thesis 
% LaTeX Template
% Version 1.43 (17/5/14)
%
% This template has been downloaded from:
% http://www.LaTeXTemplates.com
%
% Original authors:
% Steven Gunn 
% http://users.ecs.soton.ac.uk/srg/softwaretools/document/templates/
% and
% Sunil Patel
% http://www.sunilpatel.co.uk/thesis-template/
%
% License:
% CC BY-NC-SA 3.0 (http://creativecommons.org/licenses/by-nc-sa/3.0/)
%
% Note:
% Make sure to edit document variables in the Thesis.cls file
%
%%%%%%%%%%%%%%%%%%%%%%%%%%%%%%%%%%%%%%%%%

%----------------------------------------------------------------------------------------
%	PACKAGES AND OTHER DOCUMENT CONFIGURATIONS
%----------------------------------------------------------------------------------------

\documentclass[12pt, twoside]{Thesis} % The default font size and one-sided printing (no margin offsets)

\graphicspath{{Pictures/}} % Specifies the directory where pictures are stored

\usepackage[square, numbers, comma, sort&compress]{natbib} % Use the natbib reference package - read up on this to edit the reference style; if you want text (e.g. Smith et al., 2012) for the in-text references (instead of numbers), remove 'numbers' 
\hypersetup{urlcolor=blue, colorlinks=true} % Colors hyperlinks in blue - change to black if annoying
\title{\ttitle} % Defines the thesis title - don't touch this

\begin{document}

\frontmatter % Use roman page numbering style (i, ii, iii, iv...) for the pre-content pages

\setstretch{1.3} % Line spacing of 1.3

% Define the page headers using the FancyHdr package and set up for one-sided printing
\fancyhead{} % Clears all page headers and footers
\rhead{\thepage} % Sets the right side header to show the page number
\lhead{} % Clears the left side page header

\pagestyle{fancy} % Finally, use the "fancy" page style to implement the FancyHdr headers

\newcommand{\HRule}{\rule{\linewidth}{0.5mm}} % New command to make the lines in the title page

% PDF meta-data
\hypersetup{pdftitle={\ttitle}}
\hypersetup{pdfsubject=\subjectname}
\hypersetup{pdfauthor=\authornames}
\hypersetup{pdfkeywords=\keywordnames}

%----------------------------------------------------------------------------------------
%	TITLE PAGE
%----------------------------------------------------------------------------------------

\begin{titlepage}
\begin{center}

\textsc{\Large Technische Universit\"at M\"unchen}\\[.5cm] % University name
%\textsc{\Large Ludwig-Maximillians-Univesit\"at M\"unchen}\\[.5cm] % University name
\textsc{\Large Faculty of Informatics}\\[1.5cm] % University name
\textsc{\Large Bachelor's Thesis in Bioinformatics}\\[0.5cm] % Thesis type

\HRule \\[0.4cm] % Horizontal line
{\Large \bfseries PubSeq: Amino Acid-based Search Engine for MEDLINE Abstracts}\\[0.4cm] % Thesis title
{\Large \bfseries PubSeq: Aminosäuresequenz basierte Suchmachine für MEDLINE Abstrakten}\\[0.4cm] % Thesis title
\HRule \\[1.5cm] % Horizontal line
 
\begin{minipage}{0.4\textwidth}
\begin{flushleft} \large
\emph{Author:}\\
Pandu Raharja
\end{flushleft}
\end{minipage}
\begin{minipage}{0.4\textwidth}
\begin{flushright} \large
\emph{Supervisor:} \\
Prof. Dr. Burkhard Rost\\
\emph{Advisors:} \\
Dr. Guy Yachdav \\
Juan Miguel Cejuela
\end{flushright}
\end{minipage}\\[3cm]
 
% \textit{A thesis submitted in fulfilment of the requirements\\ for the degree of \degreename}\\[0.3cm] % University requirement text
%\textit{in the}\\[0.4cm]
%Technische Universit\"at M\"unchen\\Faculty of Informatics\\[2cm] % Research group name and department name
 
%{\large September 15, 2015}\\[4cm] % Date
{\large \today}\\[4cm] % Date
%\includegraphics{Logo} % University/department logo - uncomment to place it
 
\vfill
\end{center}

\end{titlepage}



%----------------------------------------------------------------------------------------
%	DECLARATION PAGE
%	Your institution may give you a different text to place here
%----------------------------------------------------------------------------------------


\Declaration{

\addtocontents{toc}{\vspace{1em}} % Add a gap in the Contents, for aesthetics

I confirm that this bachelor's thesis is my own work and I have documented all sources and material used.
 
Signed:\\
\rule[1em]{25em}{0.5pt} % This prints a line for the signature
 
Date:\\
\rule[1em]{25em}{0.5pt} % This prints a line to write the date
}

\clearpage % Start a new page

%----------------------------------------------------------------------------------------
%	QUOTATION PAGE
%----------------------------------------------------------------------------------------


\pagestyle{empty} % No headers or footers for the following pages

\null\vfill % Add some space to move the quote down the page a bit

\textit{``People usually think that progress consists in the increase of knowledge, in the improvement of life, but that isn't so. Progress consists only in the greater clarification of answers to the basic questions of life. The truth is always accessible to a man. It can't be otherwise, because a man's soul is a divine spark, the truth itself. It's only a matter of removing from this divine spark (the truth) everything that obscures it. Progress consists, not in the increase of truth, but in freeing it from its wrappings. The truth is obtained like gold, not by letting it grow bigger, but by washing off from it everything that isn't gold."}

\begin{flushright}
L. N. Tolstoy
\end{flushright}

\vfill\vfill\vfill\vfill\vfill\vfill\null % Add some space at the bottom to position the quote just right

\clearpage % Start a new page

%----------------------------------------------------------------------------------------
%	ABSTRACT PAGE
%----------------------------------------------------------------------------------------

\addtotoc{Abstract} % Add the "Abstract" page entry to the Contents

\abstract{\addtocontents{toc}{\vspace{1em}} % Add a gap in the Contents, for aesthetics

\textbf{Background}\\
In genetic research, it is imperative for biomedical researcher to stay updated on the current  state of identified proteins. It was hard -- and is getting harder, especially after widespread use of Next-Generation Sequencing (NGS) -- for researcher to keep updated on the research into protein he/she is currently investigating. This is furthermore exacerbated by the fact that existing search engines only allow querying abstracts using protein names.\\
\noindent
\textbf{Methods}\\
In this project, I present the first search engine that allows user to find all publications mentioning proteins that are similar or identical to the one he/she's interested in. To achieve this, I created a Solr Index that lists down all gene names that were mentioned in each of MEDLINE abstracts and titles. I then populated the index by scanning the whole MEDLINE corpus, tagging protein names found in title and abstract, normalizing those names into UniProt IDs and pushing the ID mentions onto Solr index. Given user's sequence query, the program runs a BLAST on the sequence and normalizes blast results to UniProt IDs. The program then retrieves articles mentioning this ID and return these to user. For the good usability I offer the whole service in a web interface available in \href{https://www.rostlab.org/}{following address}.
}

\clearpage % Start a new page

\addtotoc{Zusammenfassung} % Add the "Abstract" page entry to the Contents

\zusammenfassung{\addtocontents{toc}{\vspace{1em}} % Add a gap in the Contents, for aesthetics

\textbf{Hintergrund}\\
In genetischer Forschung ist es erzwingend, dass der/die biomedische ForscherIn mit der aktuellen Landschaft von identifizierten Protein sich st\"andig informiert. Es war schwierig -- und wird immer schwieriger sein, vor allem nach dem verbreiteten Ansatz von Next Generation Sequencing (NGS) Technoligien, um der/die Forscherin mit dem Protein von der Interesse in aktuellem Zustand zu halten. Die Tatsache, dass die aktuelle Suchmaschine von den Artikeln nur Namenbasierte Suche unterst\"utzt, hilft leider nicht weiter.\\
\noindent
\textbf{Methoden}\\
In diesem Projekt stellen wir eine Suchmaschine vor, die erlaubt den Ben\"utzer, basiert auf Aminos\"auresequenz nach den Artikeln suchen, die das Protein oder die \"Ahnliche erw\"ahnen. Um dies zu erreichen hatten wir einen Solr Index erstellt, der alle erw\"ahnte Proteine innerhalb jedes MEDLINE Artikels auflistet. Wir f\"ullen sich diesen Index in dem wir den gesamten MEDLINE Corpus durchscannen und alle Proteinname mithilfe eines NLP-Programs detektieren. Wir wurden dann diese Namen in UniProt IDs normalisieren. Diese normalisierte Namen wurden schlielich in unserem Solr Index hinzuf\"ugen. Um die Benutzbarkeit dieser Dienstleistung zu maximieren hatten wir auch eine Webschnittstelle entwurfen, die in \href{https://www.rostlab.org/}{folgender Addresse} verf\"ugbar ist.
}

\clearpage % Start a new page

%----------------------------------------------------------------------------------------
%	ACKNOWLEDGEMENTS
%----------------------------------------------------------------------------------------

\setstretch{1.3} % Reset the line-spacing to 1.3 for body text (if it has changed)

\acknowledgements{\addtocontents{toc}{\vspace{1em}} % Add a gap in the Contents, for aesthetics

First and mostly, I would like to thank Prof. Burkhard Rost for the holistic supports provided, be it through the lab infrastructures or himself personally. I would also to thank two of my advisors, Dr. Guy Yachdav and Juan Miguel Cajuela, who have in spite of their busy schedules and great distances (and time differences) patiently advised me through the project. Knowing that both are about to finish their PhD programs, I wish them all the best of luck in their future endeavors.

Also, I would like to thank Tatyana Goldberg for administrative support during my stay at the lab. Also my gratitude for Tim Karl, our awesome system administrator, who has helped us tremendously in incorporating each of the cogs in our pipeline into one coherent system. While not involved in our project personally, I would like to thank Prof. Lars Juhl Jensen of University of Copenhagen for giving us access to his tagger program. 

I would also to thank Andre Ofner for the help in validating the systems. I would also like to express my gratitude towards Robert Leaman and Zhiyong Lu from National Institue of Health. While we ended up using different implementation of normalizer in our system, their contributions during our earlier attempts in the project are not to understate.

Research wouldn't happen without grants and patrons. Therefore I would like to thank grant organizations that have contributed financial supports to the lab and its extension. I am full aware that without sufficient infrastructure and human capital support endowed by several grants, this project would be impossible to kick start and finish.

Finally I would also thank all Rostlab members and its extensions, without whom this work would all but possible.
}
\clearpage % Start a new page

%----------------------------------------------------------------------------------------
%	LIST OF CONTENTS/FIGURES/TABLES PAGES
%----------------------------------------------------------------------------------------

\pagestyle{fancy} % The page style headers have been "empty" all this time, now use the "fancy" headers as defined before to bring them back

\lhead{\emph{Contents}} % Set the left side page header to "Contents"
\tableofcontents % Write out the Table of Contents

\lhead{\emph{List of Figures}} % Set the left side page header to "List of Figures"
\listoffigures % Write out the List of Figures

\lhead{\emph{List of Tables}} % Set the left side page header to "List of Tables"
\listoftables % Write out the List of Tables

%---------------------------------------------------------
%	ABBREVIATIONS
%----------------------------------------------------------------------------------------

\clearpage % Start a new page

\setstretch{1.5} % Set the line spacing to 1.5, this makes the following tables easier to read

\lhead{\emph{Abbreviations}} % Set the left side page header to "Abbreviations"
\listofsymbols{ll} % Include a list of Abbreviations (a table of two columns)
{
\textbf{BLAST} & \textbf{B}asic \textbf{A}lignment \textbf{S}earch \textbf{T}ools\\
\textbf{MEDLINE} & \textbf{Med}ical \textbf{Li}terature Analysis Retrieval Onli\textbf{ne}\\
\textbf{NER} & \textbf{N}amed \textbf{E}ntity \textbf{R}ecognition\\
\textbf{UniProt} & \textbf{Un}iversal \textbf{Prot}ein Resources
%\textbf{Acronym} & \textbf{W}hat (it) \textbf{S}tands \textbf{F}or \\
}

%----------------------------------------------------------------------------------------
%	PHYSICAL CONSTANTS/OTHER DEFINITIONS
%----------------------------------------------------------------------------------------

%\clearpage % Start a new page

%\lhead{\emph{Physical Constants}} % Set the left side page header to "Physical Constants"

%\listofconstants{lrcl} % Include a list of Physical Constants (a four column table)
%{
%Speed of Light & $c$ & $=$ & $2.997\ 924\ 58\times10^{8}\ \mbox{ms}^{-\mbox{s}}$ (exact)\\
% Constant Name & Symbol & = & Constant Value (with units) \\
%}

%----------------------------------------------------------------------------------------
%	SYMBOLS
%----------------------------------------------------------------------------------------

%\clearpage % Start a new page

%\lhead{\emph{Symbols}} % Set the left side page header to "Symbols"

%\listofnomenclature{lll} % Include a list of Symbols (a three column table)
%{
%$a$ & distance & m \\
%$P$ & power & W (Js$^{-1}$) \\
% Symbol & Name & Unit \\

%& & \\ % Gap to separate the Roman symbols from the Greek

%$\omega$ & angular frequency & rads$^{-1}$ \\
% Symbol & Name & Unit \\
%}

%----------------------------------------------------------------------------------------
%	DEDICATION
%----------------------------------------------------------------------------------------

\setstretch{1.3} % Return the line spacing back to 1.3

\pagestyle{empty} % Page style needs to be empty for this page

%\dedicatory{For Dad, my unsung hero.} % Dedication text

\addtocontents{toc}{\vspace{2em}} % Add a gap in the Contents, for aesthetics

%----------------------------------------------------------------------------------------
%	THESIS CONTENT - CHAPTERS
%----------------------------------------------------------------------------------------

\mainmatter % Begin numeric (1,2,3...) page numbering

\pagestyle{fancy} % Return the page headers back to the "fancy" style

% Include the chapters of the thesis as separate files from the Chapters folder
% Uncomment the lines as you write the chapters

% Chapter 1

\chapter{Introduction} % Main chapter title

\label{Chapter1} % For referencing the chapter elsewhere, use \ref{Chapter1} 

\lhead{Chapter 1. \emph{Introduction}} % This is for the header on each page - perhaps a shortened title

%----------------------------------------------------------------------------------------

\section{An Easier Biomedical Research}

We'll try to present the main idea of this project in following story. Imagine you in the position as a biomedical researcher, are currently investigating some unknown enzymes that somehow were over-expressed in a patient with medical conditions. Upon some more investigating, you managed to get the sequence of several proteins. Without prior knowledge of the proteins, you would naturally BLAST the sequences and wait a little while while the BLAST is searching the sequence against your local database or some online service. Upon the results were coming, you would naturally want to check the resulting proteins one by one, at least the best matching ones. For each protein, you would want to search for articles that have dealt with this protein before.

Imagine that, instead of having to go through blasting the sequence manually and searching for articles one by one, you could just put in a sequence in a website, wait for a while and get the site returns a list of articles that mention the proteins with similar or exact sequence to the one you have. Not only you would save time and resource during the parts that were handled by website itself, you as a researcher could focus more on the substantial part of the research -- that is, finding as much essential information about the unknown protein in as little overhead as possible. Therefore, we created a web service that realizes this. In the service, user would only have to put in the sequence of unknown protein, press the search button and receive at the end a list of articles that mention proteins with identical or similar sequence to queried proteins.

With this small contribution, we hope not only to bridge the gap between sequence and knowledge discovery in biomedical research, but also give researcher more flexibility and insights in their literature research. With also ongoing feature extensions and updates, we would also hope that the service would serve more researchers with more conveniences both in medium and long run.

\section{Overview of This Thesis}

In this thesis, we will describe how we came with the idea of creating PubSeq, how we did that and what we plan in the future regarding our implementation.

In \textbf{Chapter 2}, we will discuss how bridging the knowledge gap has been attempted in the past and how our contribution would fit in the bigger picture. We also discuss some of the methods that are relevant in our project. Also, we would look into how our project builds upon existing knowledge and technology.

\textbf{Chapter 3} introduces the system as a whole. How we organize the sub-components together. We would also skim through the technological side of the projects here, while keeping the reader aware of the bigger picture. We would discuss our rationale behind selecting some of technology stacks that we used. All the while,we would also show some the visual examples from our component here. By the end of the chapter it is hoped that the reader would understand how each single component interacts with others within our system.

\textbf{Chapter 4} explains in detail our Tagging Pipeline. Here the reader would see how the MEDLINE corpus would be processed, annotated and pushed onto Solr Index.

\textbf{Chapter 5} explains our storage technology, the Solr index. Besides the technology itself, we would explain how we configure and structure our index that would fit the resulting data from Tagging Pipeline.

\textbf{Chapter 6} investigates our web server component. We would first present the technological stacks that were used in this component and other specifications. We would then present the program from the perspective of end user. We would show how convenient would that be for a researcher to use our application, which would make our case for value proposition of PubSeq search engine.

\textbf{Chapter 7} covers quantitative measurement of the quality of our website. We would focus mostly on how our system performs, especially with regards to the sensitivity and specificity. We would focus mostly on the the quality of protein tagging within our data (see Chapter \ref{Chapter5} for detail). We would also muse on how our system would have an edge over similar UniProt ID-based abstract search service provided by UniProt \citep{uniprot2008universal} \citep{Uniprot2011} \citep{UniprotOnline}.

%\textbf{Chapter 6} explains our update and maintenance design for the website. Here the reader would be aware on how we attempt to make our website up-to-date to the latest protein landscape. We would, again, delve into how this is realized within our system.

\textbf{Chapter 8} covers our conclusion of the system so far. There we presented our own ideas on how the system has achieved so far. We would then briefly mention how we are going to proceed in the near future.

%----------------------------------------------------------------------------------------

% Chapter Template

\chapter{Background} % Main chapter title

\label{Chapter2} % Change X to a consecutive number; for referencing this chapter elsewhere, use \ref{ChapterX}

\lhead{Chapter 2. \emph{Background}} % Change X to a consecutive number; this is for the header on each page - perhaps a shortened title

This chapter introduces the concepts and techniques that are relevant throughout this thesis. First, the concept of similarity search, especially the two software suite FASTA and BLAST would open our chapter. And then, we would introduce various contemporary concepts in bioinformatics and bioinformatics-related infrastructure such as UniProt and MEDLINE. Additionally, we would introduce the concept of named entity recognition (NER) within the field of Natural Language Processing and how it would be relevant for us. Finally we would see how our project relates to previous works in similar topics and how it would improve, provide alternative or give additional insight to them.

%----------------------------------------------------------------------------------------
%	SECTION 1
%----------------------------------------------------------------------------------------

\section{FASTA and BLAST}

As the title of this thesis already conveyed, the main idea of this project is to bridge the accessibility and knowledge gap between sequence and the main source of knowledge and reference of previous discoveries -- a vast corpora of publications in natural sciences -- through a modern search engine. Given a sequence of amino acids, it would be impossible for a human to directly identify directly the protein, let alone the characteristics and the functions and the characteristics of the protein.

Several attempts on bridging one component of the gap, specifically between sequence and other known sequences, was done in eighties and earlier nineties. In 1981, Smith and Walterman published the algorithm computing complete local sequence alignment, which was further improved by Gotoh in 1982 \citep{gotoh1982improved} and Altschul (Altschul and Erickson, 1986 \citep{altschul1986optimal}). This was however deemed too slow, especially if used for the purpose of one-against-all search, which was heavily (and still is) used for sequence-based knowledge discovery in biomedical research.

In 1985, Lipman and Pearson published the first paper mentioning the DNA and protein sequence alignment program FASTA \citep{Lipman85}. During the first publication, FASTA was designed and intended to search for similar protein sequences. It takes a sequence of amino acids and searches against entries within a corresponding database by using local sequence alignment to find similar sequences. In general, FASTA takes four steps in computing three scores that characterize sequence similarity \citep{Pearson19905}.

\begin{enumerate}
\item Finding identify regions with high density of sequence identities and pair identities between two sequences. FASTA achieved a fast computation in this step by using a look up table, a map that describes for each character where it appears within sequence. In conjunction with the lookup table, FASTA also uses the diagonal method to find 
all regions of similarity between the two sequences, counting matches and penalizing for intervening mismatches. This diagonal could be visually seen in two sequence alignment as series of matches ('dots') in match matrix between two sequences.

\item Rescanning of the 10 regions with highest sequence identities using PAM250 matrix. PAM250 matrix refers to assumed point accepted mutation (PAM) matrix after 250 mutations, which is basically the 250-th power of initial PAM matrix. The probability of each entry within PAM matrix was acquired from analysis of phylogenetic trees (Dayhoff, 1978 \citep{Dayhoff1978model}).

\item Annealing of both ends of alignment and calculating similarity score is the sum of the joined initial regions minus a penalty (usually 20) for each gap \citep{Pearson19905}.

\item Construction of optimal alignment using Needleman-Wunsch Algorthm \citep{needleman1970general} on the best matching region. The program would then return the similarity score of this alignment along with the best score from step 2 and 3.
\end{enumerate}

In 1988, Pearson and Lipman improved the software by adding support and improvement, among others, for nucleic acid similarity search, translated nucleic acid search \citep{PearsonLipman88}. This allowed researchers to do trans-domain search between nucleic and amino acids.

Further down the road, in 1990, Altschul et al. published the Basic Alignment Research Tool \citep{Altschul90}, better known in its acronym as BLAST. The algorithm, like FASTA, is based on heuristics search and is structured in similar manner to BLAST 

%Nowadays, FASTA not only had added support for amino acids sequence similarity search, but also 

%----------------------------------------------------------------------------------------
%	SECTION 2
%----------------------------------------------------------------------------------------

\section{Previous Works}

There are several previous works that we are aware of that try to tackle similar problems.  
% Chapter Template

\chapter{Organizations and Components} % Main chapter title

\label{Chapter3} % Change X to a consecutive number; for referencing this chapter elsewhere, use \ref{ChapterX}

\lhead{Chapter 3. \emph{Organizations and Components}} % Change X to a consecutive number; this is for the header on each page - perhaps a shortened title

This chapter enumerates the components constituting the PubSeq system and explains how each in principle functions. There would also be technicalities of the programs and rationales on how each of the component is used and implemented.

%----------------------------------------------------------------------------------------
%	SECTION 1
%----------------------------------------------------------------------------------------

\section{Introduction}

\label{sec:Chap3Intro}

In the most general term, there are three main components that build up the PubSeq environment. The three components could be represented as questions:

\begin{itemize}
\item How to \textit{create} the data?
\item How the data is going to be \textit{stored}?
\item How the user is going to \textit{retreive} the data?
\end{itemize}

We \textit{create} the data in which we process the raw data from our source into indexable entry data. We then \textit{store} the data in scalable manner for the user to use. Finally, we will facilitate how a user could \textit{retreive} our data.

We formalize this concept further by creating three main components with each represents the answer of the question before:

\begin{itemize}
\item \textbf{PubSeq Tagging Pipeline} (\textbf{Tagging Pipeline} for short) addresses the first question. In this pipeline we would process the data from its main source, the MEDLINE corpus which is updated daily as XML file, onto indexable input file containing list of UniProt annotations in the abstracts, among others.
\item \textbf{PubSeq Solr Index} (\textbf{Solr Index} for short) addresses storage issue. All processed data would then in an open source enterprise search platform, Solr \citep{smiley2015apache}.
\item \textbf{PubSeq Web Server} (\textbf{Solr Server}) addresses the issue of content delivery to end user. Here we would explain the program in more technical manner. We would also convey how the program would see in the perspective of end user -- that is how the program runs as user proceeds on using the search engine, later in Chapter \ref{Chapter4}.
\end{itemize}

%----------------------------------------------------------------------------------------
%	SECTION 2
%----------------------------------------------------------------------------------------

\section{PubSeq Persistent Components}

To understand how the components were structured, I think it is better for the reader to first get to know how the persistent components, that is, the components that are running around-the-clock interact:

\begin{figure}[htbp]
  \centering
    \includegraphics[width=6in]{Figures/solr_graph_main.png}
    \rule{35em}{0.5pt}
  \caption[An Overview of how the 'Persistent Components' of PubSeq environment interacts.]{An overview of how the Persistent Components -- the components that are running around the clock -- interact with each other. Here we can see two of three main components explained before: Solr Index, Solr Server (and its rendered web app instance, PubSeq Web Page) which is also connected with RostLab Cluster via SunGrid Engine (SGE). Compare this diagram with spatially differentiated diagram of the system with Virtual Machine shown (Figure \ref{fig:PubSeqVM} in Chapter 6)}
  \label{fig:ComponentInteraction}
\end{figure}

Here we can see three main components within the interaction environment: Solr Index, PubSeq Server, PubSeq web page and RostLab Cluster (SunGrid Engine).

\begin{itemize}
\item \textbf{PubSeq Web Page} is rendered by PubSeq server upon GET request on PubSeq home path and communicates through HTTP protocols to PubSeq server. There are currently POST and GET methods that are launched by this page onto the server. Beside the server, the PubSeq web server, the web page also presents some links to outside world, most notably to PubMed web interface \citep{MELDINEWeb}. The linkage to PubMed web interface was embeded in each of the results of the query within PubSeq (see Chapter \ref{Chapter4} for details)
\item \textbf{PubSeq Web Server} is server component of our system. It handles HTTP requests addressed to the web path and renders PubSeq web page. Internally it processes both queries and results from both web page and server reply. The server also communicates to RostLab's clusters for submitting BLAST query for a given sequence.
\item \textbf{PubSeq Solr Index} contains the whole indexing of MEDLINE and proteins mentions within article. While the main mechanism of updating the index will be explained thoroughly in later chapter, it is important to know for reader that the index only communicates via HTTP \citep{smiley2015apache}. As thus, user can check on the machine that the server runs on, the sample content of the Solr server by doing curl followed by the query. For more details see dedicated chapters bellow.
\item We utilize \textbf{RostLab Cluster} for BLASTing input sequence given by the user.
\end{itemize}

%----------------------------------------------------------------------------------------
%	SECTION 3
%----------------------------------------------------------------------------------------

\section{Web Interaction}

We model our web interaction in following time-dependent sequence diagram \citep{rumbaugh2004unified} on Figure \ref{fig:SequenceDiagram}.

\begin{figure}[htbp]
  \centering
    \includegraphics[width=6in]{Figures/sequence_diagram.png}
    \rule{35em}{0.5pt}
  \caption[A Sequence Diagram modeling of PubSeq web interaction.]{Time-dependent sequence diagram of PubSeq web interaction. The y-axis represents the time (from top to bottom) and each of the columns in on the x-axis represents components that interact with each others. Note that the in the third column, there are two entities that interact with Web Server in distinct time spans.}
  \label{fig:SequenceDiagram}
\end{figure}

Here we see how four main persistent components of PubSeq interact. First, the user opens the PubSeq page. This will spawn an instance of PubSeq web page. Upon inserting some sequence and pressing query button, the web page would then submit the initial HTTP request \citep{fielding1999hypertext} onto the server (first \textbf{POST sequence}). Our Node.js server \citep{tilkov2010node} would then handle the query and create a script and input file containing aforementioned sequence. It would then submit the query onto RostLab cluster through qsub (\textbf{Submit job}) \citep{gentzsch2001sun} and return first response containing message informing the web client that the job has been submitted (\textbf{RESPONSE submitted}). The web client would then re-check the server once in a while (ca. 10 seconds) to find out whether the result of the query has been created (\textbf{POST check} and \textbf{RESPONSE running}). Once the job has been completed the results would be saved in a pre-defined location using pre-defined names (see later chapter for more details). During the next iteration of checking routine from web app, if the output is already there, the request handler would then parse the output file and prepare a Solr query that would be used for the sequence. This query would then be submitted onto Solr client (\textbf{Query Solr}). The query response would then be forwarded to web page in the RESPONSE massage (\textbf{RESPONSE results}) and the results would be shown to the user. Had the user have to update the results (mostly by moving between result pages), an update POST would be initiated (\textbf{POST sequence}). This update POST contains prepared statement that doesn't require another BLAST, and thus would be directed toward Solr index directly (second \textbf{Query Solr}). Just like initial Solr response, the resulting query would be returned to user. As long as user doesn't leave the page or query new sequence, this iteration could be done as many times as possible.

%----------------------------------------------------------------------------------------
%	SECTION 4
%----------------------------------------------------------------------------------------

\section{Tagging Pipeline}

Tagging pipeline refers to the process of creating and updating the Solr index that would be used for the server to search for articles containing protein mentions. In this pipeline, a set of documents from MEDLINE database would sequentially processed and finally be pushed into our index. The whole process is done in periodic/one-off basis. The schematic representation with each single step obscured can be seen in Figure \ref{fig:TaggingPipelineBroad}.

\begin{figure}[htbp]
  \centering
    \includegraphics[width=6in]{Figures/solr_tagging_pipeline.png}
    \rule{35em}{0.5pt}
  \caption[Schematic representation of PubSeq Tagging Pipeline.]{Schematic diagram representing the broad process of Solr Tagging Pipeline. On the left side the input files, MEDLINE abstracts in XML formats, would be processed sequentially until the data is ready to be pushed onto the Solr index. We deliberately highlighted NER Tagging from within the pipeline to emphasize its importance in our pipeline.}
  \label{fig:TaggingPipelineBroad}
\end{figure}

Here we see that the process consists of smaller processes. As already said, there could be two way of running this process: one-off and periodic update. While one-off Tagging Pipeline could be done on RostLab's \textbf{jobtest}/\textbf{jobtest2} or similar physical servers within RostLab infrastructure, the periodic update process should be only done via SunGrid Engine call scheduled in crontab \citep{keller1999take}.

We deliberately showed the NER Tagger process within this Pipeline to reader to emphasize its importance. Lars Juhl Jensen was kind enough to give us his program that we use to run on our pipeline. There would be more details on both our NER Tagger and Tagging Pipeline in later chapter(s) but for now we would focus on big picture detail.
% Chapter Template

\chapter{PubSeq tagging Pipeline} % Main chapter title

\label{Chapter4} % Change X to a consecutive number; for referencing this chapter elsewhere, use \ref{ChapterX}

\lhead{Chapter 4. \emph{PubSeq tagging PipelinePipeline}} % Change X to a consecutive number; this is for the header on each page - perhaps a shortened title

%----------------------------------------------------------------------------------------
%	SECTION 1
%----------------------------------------------------------------------------------------

\section{Introduction}

In this chapter, we would discuss various steps belonging to Tagging Pipeline. Following main steps belong to the Tagging Pipeline:

\begin{enumerate}
\item \textbf{Formatting} downloaded MEDLINE abstract into input files that are compliant with Lars' NER Tagger. \label{itm:TaggingStep1} (1)
\item \textbf{Named Entity tagging} done by Lars NER Tagger. \label{itm:TaggingStep2} (2)
\item \textbf{Post-processing} of the results and \textbf{preparation} for the entry into Solr index. \label{itm:TaggingStep3} (3)
\item \textbf{Updating} of results onto Solr Index. \label{itm:TaggingStep4} (4)
\end{enumerate}

The processes are then further divided into several single programs:

\begin{itemize}
\item \texttt{XMLAbstractsFormatter.java}
\item \texttt{tagcorpus.cxx}
\item \texttt{AnnotationBackmapper.java}
\item \texttt{Annotater.java}
\item \texttt{StatisticsUtils.java}
\item \texttt{IndexerNew.java}
\item \texttt{SolrUpdater.java}
\end{itemize}

The division of the whole pipeline into smaller tasks are reasoned through following arguments:
\begin{itemize}
\item The NER Tagger \ref{itm:TaggingStep2} was developed in C++ while we would implement the rest of pipeline (\ref{itm:TaggingStep1}, \ref{itm:TaggingStep3} and \ref{itm:TaggingStep4}) in Java. This means that pre- and post-tagging procedures would have to be implemented separately. Also since the NER Tagger wasn't written by us and therefore support would be very lacking, we would rather leave the NER Tagger as it is.
\item Related to previous argument: Solr is implemented in Java and its most comprehensive API (written by the developers of the Solr themselves) is written in Java \citep{grainger2014solr}. Therefore using Java in \ref{itm:TaggingStep4} would be almost necessary for various convenience reasons.
\item The pipeline generally is very memory intensive. During the process, several maps that each would gulp easily teens of gigabyte of memory are utilized. Therefore each step that utilizes such huge maps are all separated into one single routine.
\item Dividing the pipeline into smaller components would make it easy to debug, since each routine easily takes several minutes if not hours to run. However this also makes expanding features within the Tagging Pipeline more difficult.
\end{itemize}

Besides processes mentioned above, there is one process that doesn't exactly belong to Tagging Pipeline but is closely coupled and synchronized with it: downloading and storing of MEDLINE abstracts. Both download and maintenance of MEDLINE corpus and the Tagging pipeline would be covered in following sub-chapters.

%----------------------------------------------------------------------------------------
%	SECTION 2
%----------------------------------------------------------------------------------------

\section{MEDLINE Abstracts}

Downloaded MEDLINE abstracts could be found at /mnt/project/rost\_db/medline. There are two sub directories found in the path that are relevant to PubSeq: basline and update. Baseline contains all MEDLINE abstracts that were downloaded up until 24 Jan 2015. Update contains all MEDLINE abstracts that were downloaded afterwards. The MEDLINE abstracts are stored in XML files. The XML format for MEDLINE abstracts currently follow the definition set by NLM (Document Type Definitions or DTD for MEDLINE/PubMed, see here or download here). Please refer to the data definition for further property of the corpus.


It had been agreed upon that the MEDLINE will be updated every day at 8 am server time. 
% Chapter Template

\chapter{Maintenance and Updates} % Main chapter title

\label{Chapter5} % Change X to a consecutive number; for referencing this chapter elsewhere, use \ref{ChapterX}

\lhead{Chapter 5. \emph{Maintenance and Planned Features Expansion}} % Change X to a consecutive number; this is for the header on each page - perhaps a shortened title 
% Chapter Template

\chapter{PubSeq Web Service} % Main chapter title

\label{Chapter6} % Change X to a consecutive number; for referencing this chapter elsewhere, use \ref{ChapterX}

\lhead{Chapter 6. \emph{PubSeq Web Service}} % Change X to a consecutive number; this is for the header on each page - perhaps a shortened title

In this chapter we address the third question posed in Section \ref{sec:Chap3Intro}: \textit{How the user is going to \textit{retreive} the data?}. First, we would define our Virtual Machine environment on which our system runs. We would then delve into the structure of our Node.js server, how it initializes a web page and how it would interact with our service. Finally we would see how a user would generally use PubSeq, from giving in sequence to receiving and navigating over the results.


%----------------------------------------------------------------------------------------
%	SECTION 1
%----------------------------------------------------------------------------------------
\section{PubSeq Virtual Machine}

Two main components of PubSeq: PubSeq Solr Index and PubSeq Node.js server lay within a defined Virtual Machine within Rostlab internal network. Once a user opens a path that is directed to our Solr web server, the request would first land on Rostlab.org web server. Rostlab.org web server would then forward this request further to our PubSeq Node.js web server. The service would then render the page that the user will use to communicate with our server. All requests and responses between web page and PubSeq Node.js server would be proxied through Rostlab.org apache server (see Figure \ref{fig:PubSeqVM}). Our Virtual Machine is accessible internally via an assigned IP address and could also access other components within Rostlab network such as the SunGrid Engine (SGE). This IP address is however, not accessible from outside world. This guarantees that our Virtual Machine, particularly the Solr Index, which communicate via HTTP, couldn't be accessed from outside the world and thus prevents an unwanted disruption such as database injection.

To further ensure consistency of our Virtual Machine, we create a snapshot of the the VM once every day. Out of daily snapshots, we would keep the snapshots form the last two days. This way, we can always roll back our VM in case something would happen on it.

\begin{figure}[htbp]
    \includegraphics[width=6in]{Figures/pubseq_vm.png}
    \rule{35em}{0.5pt}
  \caption[Schematic representation of PubSeq systems with the Virtual Machine environment shown.]{Schematic representation of PubSeq systems with the Virtual Machine environment shown. Here we see the relative position and content of our Virtual Machine within Rostlab local area network (LAN). Rostlab SunGrid Engine (SGE) cluster and most other constituents within Rostlab networks are accessible from Virtual Machine \textit{vice versa}.}
  \label{fig:PubSeqVM}
\end{figure}

\subsection{VM Specifications}

Following are the specifications of our Virtual Machine:

\textbf{Kernel:}

\begin{lstlisting}[breaklines]
Linux pubseq-web.rostclust 3.16.0-4-amd64 #1 SMP Debian 3.16.7-ckt11-1 (2015-05-24) x86_64 GNU/Linux
\end{lstlisting}

\textbf{Distribution:}
\begin{lstlisting}[breaklines]
Disribution:		Debian GNU/Linux 8.1 (jessie)
Release:		8.1
Codename:		jessie
\end{lstlisting}


\textbf{Processors:}
\begin{lstlisting}[breaklines]
Core:			1
CPU MHz:		2599.998
BogoMIPS:		5199.99
CPU RAM GB:		6
Architecture:		x86_64
\end{lstlisting}

Here we see that our VM has relatively low RAM and number of core compared to other constituents within the network \footnote{One typical node in Rostlab's cluster has about 32 GB RAM and 12 cores}. The reason for this is because computationally expensive computations within our system, PubSeq Tagging Pipeline and BLAST, are always delegated to Rostlab SunGrid Engine (SGE). The two components that persistently run on our VM, Node.js Server and Solr, don't take a lot of memory and computing power. A lightweight Solr instance usually takes about 500 MB memory to run -- in our case, we only limit our memory use use to 1 GB RAM. Node.js is elastically implemented -- that is, it only requires memory that it needs. This means that in a given time, when there is no request, memory usage is nothing but negligible.

%----------------------------------------------------------------------------------------
%	SECTION 2
%----------------------------------------------------------------------------------------
\section{PubSeq Web Server}

In this project, we used Node.js for server side networking and scripting purpose. Node.js is a runtime environment for applications written in JavaScript. Unlike conventional JavaScript which runs on webpage within web browser engine, Node.js enables JavaScript application to run in command line environment. It provides event-driven and lock-free I/O API \citep{nodejs}, which makes it suitable for real-time web application. Node.js achieves this by utilizing the so-called Event Loop (see Figure \ref{fig:NodeJS}). 


Node.js is based on V8 Virtual Machine, which unlike other more traditional JavaScript VM, compiles JavaScript to native machine code instead of interpreting the code and then compiling it \citep{v8javascript}. All these characteristics make Node.js suitable for running a web application that handles thousands of concurrent connections with minimal overhead possible. We also chose Node.js as our web server environment since it allows compatibility with regard to data communication between web server and client, since native JSON would be used to communicate between the two. Specifically for Node.js, we use Express framework to implement our server, which is the most commonly used web service framework in Node.js \citep{expressjs}.

\begin{figure}[htbp]
    \includegraphics[width=6in]{Figures/nodejs.png}
    \rule{35em}{0.5pt}
  \caption[Schema of Node.js's Processing Model.]{Node.js internal processing model. Upon the arrival of HTTP request in Node.js server, Event Loop, which is implemented as single thread, passes the request onto worker threads while also adding callback onto its stack. Upon finishing submitted job, callback function will be called to notify Event Loop. It would then return job's results to the requesting client. By avoiding spawning thread for every request, Node.js avoids the overhead that could occur in server that handles thousands of request in a given time. This makes it appropriate for multi-users web application. Figure adopted from (Stannard, 2011 \citep{stannard2011intro}) with modifications.}
  \label{fig:NodeJS}
\end{figure}

\subsection{Components}

There are two main components in our Node.js server: \textbf{app.js} and web page files.

\textbf{app.js}\footnote{See Appendix \ref{sec:AppJSPath} for the paht to app.js within the repository.} is the main entry in our server. It contains components that are needed to make the server running such as:

\begin{itemize}
\item Express.js application definition and initiation \footnote{See \href{http://expressjs.com/4x/api.html}{\texttt{http://expressjs.com/4x/api.html}} (accessed 26/08/2015) for Express.js API.}.
\item HTML rendering engine definition (see \textbf{Web Page Files} bellow).
\item Additional prototypical String functions such as hash code creating function  and \texttt{startsWith()} \footnote{Note that \texttt{startsWith()} will be implemented in upcoming ECMAScript 2015 (ES6) Standard, see \href{https://developer.mozilla.org/en/docs/Web/JavaScript/Reference/Global_Objects/String/startsWith}{\texttt{https://developer.mozilla.org/en/docs/Web/JavaScript/Reference/Global\_Objects/String/startsWith}} (accessed 26/08/2015).}
\item \texttt{GET}, \texttt{POST} handlers for each of available pages, if such method is defined within the page.
\item Several utility functions.
\end{itemize}

\textbf{Web Page Files} consist of pages that are available within the web application environment. Each page is written in Jade syntax, which makes it easier for user to write complex HTML file \footnote{\href{http://naltatis.github.io/jade-syntax-docs/}{\texttt{http://naltatis.github.io/jade-syntax-docs/}, accessed 26/06/2015.}}. Upon \texttt{GET} request for a page, \textbf{app.js} would render the page file using the HTML rendering engine for jade. This engine would parse the \texttt{.jade} file and create proper HTML file, which would then be returned to client.

\subsection{index.html}

Of most importance is the \textbf{index.html} (which was avaiable as \textbf{index.jade} prior rendering process) among web pages that are avaiable within the web interface environment. The web page contains the query interface that would communicate. 
% Chapter Template

\chapter{Results and Validations} % Main chapter title

\label{Chapter7} % Change X to a consecutive number; for referencing this chapter elsewhere, use \ref{ChapterX}

\lhead{Chapter 7. \emph{Results and Validations}} % Change X to a consecutive number; this is for the header on each page - perhaps a shortened title
 

%----------------------------------------------------------------------------------------
%	THESIS CONTENT - APPENDICES
%----------------------------------------------------------------------------------------

\addtocontents{toc}{\vspace{2em}} % Add a gap in the Contents, for aesthetics

\appendix % Cue to tell LaTeX that the following 'chapters' are Appendices

% Include the appendices of the thesis as separate files from the Appendices folder
% Uncomment the lines as you write the Appendices

% Appendix A

\chapter{PubSeq Paths and Source URLs} % Main appendix title

\label{AppendixA} % For referencing this appendix elsewhere, use \ref{AppendixA}

\lhead{Appendix A. \emph{PubSeq Paths and Source URLs}} % This is for the header on each page - perhaps a shortened title

%----------------------------------------------------------------------------------------
%	SECTION 1
%----------------------------------------------------------------------------------------

\section{Source Codes}

There are several source codes that are relevant to our project:

\begin{itemize}
\item \href{https://rostlab.informatik.tu-muenchen.de/gitlab/gyachdav/pubseq-crawler}{pubseq-crawler} contains the tagging ('crawler') pipeline of our project. It also contains several scripts that are used in this project (most notably \texttt{annotate\_new.sh} and \texttt{maintenance.sh}.
\item \href{https://rostlab.informatik.tu-muenchen.de/gitlab/gyachdav/pubseq-frontend}{pubseq-frontend} contains the node.js implementation of our project.
\end{itemize}

%----------------------------------------------------------------------------------------
%	SECTION 2
%----------------------------------------------------------------------------------------

\section{Project Location}

Generally the project will be located at \texttt{/mnt/project/pubseq/} within Rostlab server. The directory is further divided into following categories:

\begin{verbatim}
- /mnt/project/pubseq
|- index_input_docs
|- log
|- named_entity_tagger
|- programdata
|- rundata
|- solr_index
|- scripts
|- stats 
\end{verbatim}

The overview of each of directories is as follow:

\begin{itemize}
\item \texttt{index\_input\_docs} contains the files that would be indexed onto Solr. In other words, this directory contains all files that were created during Tagging Pipeline. During the last step of Tagging Pipelines, the program \texttt{SolrUpdater.jar} would point at this directory and index files that match certain pattern of file name.
\item \texttt{log} contains logs from Tagging Pipeline.
\item \texttt{named\_entity\_tagger} contains the \textbf{Lars Tagger component} of the Tagging Pipeline.
\item \texttt{programdata} contains several program-related tab-separated files that would be used and/or produced during the Tagging Pipeline. Note that program-related data is different from run-related data (which are located at \texttt{rundata} in which program-related data remain mostly constant across all runs while run-related data were created during one of the steps within the Tagging Pipeline. 
\item \texttt{rundata} contains interim results from Tagging Pipeline and other Tagging-related data. Since Tagging Pipelines consist of multiple programs with each taking input and writing output, it is inevitable that there would be several interim values. The Tagging Pipeline is designed to write standardized in-between values. While the files would not be removed upon completion of one Tagging Pipeline run, they would be overwritten during the next run. The interim values wouldn't be backed up in some consistent environment.
\item \texttt{solr\_index} contains the Solr directory for the project. For readers who are not familiar with Solr, assuming it as "NoSQL Database" would be sufficient. Solr index is self-contained in the way that the only thing that is needed for it to run properly is its own directory (and JDK).
\item \texttt{scripts} contain scripts that would be run for various purposes. The most essentials of all scripts are the \texttt{annotate\_new.sh} and \texttt{maintenance.sh}. \texttt{annotate\_new.sh} wraps the whole Tagging Pipeline process and calls sequentially each component within the pipeline. \texttt{maintenance.sh} contains the Maintenance Pipeline and like \texttt{annotate\_new.sh} calls each component within Maintenance Pipeline sequentially. Both Tagging and Maintenance Pipelines would be further described in later chapters.
\item \texttt{stats} is deprecated or might not be necessary for the whole process. It contains some descriptive statistics from the last run of Tagging Pipeline. For each Tagging Pipeline, some set of statistic files were created that more or less describe the nature of the run.
\end{itemize}2
%% Appendix A

\chapter{Figures and Tables} % Main appendix title

\label{AppendixB} % For referencing this appendix elsewhere, use \ref{AppendixA}

\lhead{Appendix B. \emph{Figures and Tables}} % This is for the header on each page - perhaps a shortened title

\begin{sidewaysfigure}[ht]
    \includegraphics{Figures/tagging_pipeline_complete.png}
    \caption[Overview of PubSeq Tagging Pipeline with all essential programs showed as nodes and important input/output files shown]{Full-sized Overview of PubSeq Tagging Pipeline.}
    \label{fig:PubSeqTaggingFull}
\end{sidewaysfigure}

\begin{figure}

\begin{minipage}{.5\linewidth}
  \centering
  \subfloat[]{\label{AB1:a}\includegraphics[width=3in]{Figures/eval/appendix/Micro_BC2.png}}
  \end{minipage}%
  \begin{minipage}{.5\linewidth}
  \centering
  \subfloat[]{\label{AB1:b}\includegraphics[width=3in]{Figures/eval/appendix/Micro_Craft.png}}
  \end{minipage}\par\medskip
  \begin{minipage}{.5\linewidth}
  \centering
  \subfloat[]{\label{AB1:c}\includegraphics[width=3in]{Figures/eval/appendix/Micro_IDP4.png}}
  \end{minipage}%
  \begin{minipage}{.5\linewidth}
  \centering
  \subfloat[]{\label{AB1:d}\includegraphics[width=3in]{Figures/eval/appendix/Micro_LocText.png}}
  \end{minipage}\par\medskip

  \caption[Overview of the highest F-Scores in exact, average and fuzzy tagging evaluation]{Overview of detailed F-Scores reached in exact, average and fuzzy tagging, sorted by the score achieved in averaged matching. \ref{AB1:a}: BioCreative II. \ref{AB1:b}: Craft. \ref{AB1:c}: IDP4. \ref{AB1:d} LocText. Adopted from (Ofner, 2015 \citep{ofner2015evaluation})}
\label{fig:AB1}
\end{figure}

\begin{figure}

\begin{minipage}{.5\linewidth}
  \centering
  \subfloat[]{\label{AB2:a}\includegraphics[width=3in]{Figures/eval/appendix/NormalizationMention_Craft.png}}
  \end{minipage}%
  \begin{minipage}{.5\linewidth}
  \centering
  \subfloat[]{\label{AB2:b}\includegraphics[width=3in]{Figures/eval/appendix/NormalizationMention_LocText.png}}
  \end{minipage}%

  \caption[Overview of detailed F-Scores reached for mention based normalization]{Overview of detailed F-Scores reached for mention based normalization. \ref{AB2:a}: Craft. \ref{AB2:b}: LocText. Adopted from (Ofner, 2015 \citep{ofner2015evaluation})}
\label{fig:AB2}
\end{figure}

\begin{figure}

  \begin{minipage}{.5\linewidth}
  \centering
  \subfloat[]{\label{AB3:a}\includegraphics[width=3in]{Figures/eval/appendix/NormalizationDocument_BC3.png}}
  \end{minipage}%
  \begin{minipage}{.5\linewidth}
  \centering
  \subfloat[]{\label{AB3:b}\includegraphics[width=3in]{Figures/eval/appendix/NormalizationDocument_Craft.png}}
  \end{minipage}\par\medskip
  \centering
  \subfloat[]{\label{AB3:c}\includegraphics[width=3in]{Figures/eval/appendix/NormalizationDocument_LocText.png}}

  \caption[Overview of detailed F-Scores reached for mention based normalization.
Standard errors are indicated for each F-Score]{Overview of detailed F-Scores reached for mention based normalization. \ref{AB3:a}: BioCreative III. \ref{AB3:b}: Craft. \ref{AB3:c}: LocText. Adopted from (Ofner, 2015 \citep{ofner2015evaluation})}
\label{fig:AB3}
\end{figure}

%\input{Appendices/AppendixC}

\addtocontents{toc}{\vspace{2em}} % Add a gap in the Contents, for aesthetics

\backmatter

%----------------------------------------------------------------------------------------
%	BIBLIOGRAPHY
%----------------------------------------------------------------------------------------

\label{Bibliography}

\lhead{\emph{Bibliography}} % Change the page header to say "Bibliography"

\bibliographystyle{unsrtnat} % Use the "unsrtnat" BibTeX style for formatting the Bibliography

\bibliography{Bibliography} 

% The references (bibliography) information are stored in the file named "Bibliography.bib"

\end{document}  